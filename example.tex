\documentclass{beamer}

\usetheme{simple}

\usepackage{lmodern}
\usepackage[scale=2]{ccicons}

% TODO: 
%   position adjustement
%   change colours
%       

% Watermark background (simple theme)

\setwatermark{\includegraphics[height=8cm]{logga.png}}

\title{Optik}
\subtitle{}
\date{\today}
\author{Magnus Wass}
\institute{\url{magnus.wass@nykopingsgymnasium.com}}

\begin{document}

\maketitle

\begin{frame}{Reflektion}
  \framesubtitle{ergo 264-68}





      \begin{itemize}
        \item \alert{Reflektionslagen} s\"ajer att reflektionsvinkeln 
        och infallsvinkeln \"ar lika stora
        \item Kom ih\aa g att vinklarna m\"ats fr\aa n \alert{normalen}
        \item \alert{Konkava speglar} ger en f\"orminskad, \emph{reell}, uppochnerv\"and 
        bild om f\"orem\aa let \"ar n\"ara spegeln och en f\"orstorad, \emph{virtuell} och
        r\"attv\"and bild annars
        \item \alert{Konvexa speglar} ger alltid en f\"orminskad,\emph{virtuell} och
        r\"attv\"and bild
      \end{itemize}


\end{frame}





\begin{frame}{Brytning}
\framesubtitle{ergo 269-74}
\begin{itemize}
\item \alert{Fermats princip} inneb\"ar att ljuset f\"oljer den snabbaste v\"agen mellan tv\aa punkter och leder till
 \item \alert{brytningslagen} $n_1\cdot \sin v_1=n_2\cdot \sin v_2$
 \item h\"ar \"ar det \"annu viktigare att komma ih\aa g att vinklarna m\"ats fr\aa n normalen!
 \item Ett mediums \alert{brytningsindex(n)} definieras som ljushastigheten i vakuum(c) delat med ljushastigheten i mediet(v), dvs. $n=\frac{c}{v}$.Eftersom $c \geq v$ g\"aller \"aven $n \geq 1$ 
\end{itemize}
\end{frame}






\begin{frame}{Prisma och linser}
\framesubtitle{ergo 275-83}

  \begin{itemize}
    \item Synligt ljus har v\aa gl\"angder i intervallet $400 nm$(violett)-$700 nm$(r\"ott)
    \item Brytningsindex minskar med \"okande v\aa gl\"angd
    \item F\"orem\aa ls f\"arg beror p\aa vilka v\aa gl\"angder som reflekteras
    \item vita f\"orem\aa l reflekterar alla (synliga) v\aa gl\"angder, svarta absorberar allt synligt ljus
    \item solen \"ar en svart kropp
    \item En \alert{konvex lins} bryter parallella(med optiska axeln) str\aa lar mot fokus och
    \item str\aa lar fr\aa n fokus bryts parallellt med optiska axeln
    \item En \alert{konkav lins} bryter parallella str\aa lar s\aa att de ser ut att komma fr\aa n fokus och 
    \item str\aa lar som ser ut att vara p\aa v\"ag mot fokus bryts parallella

  \end{itemize}


\end{frame}




\begin{frame}{Gauss linsformel och optiska instrument}
\framesubtitle{ergo 283-93}

  \begin{itemize}
    \item \alert{Gauss linsformel} lyder $\frac{1}{f}=\frac{1}{a}+\frac{1}{b}$
    d\"ar 
\begin{itemize}
\item f \"ar avst\aa ndet mellan lins och br\"annpunkt(kallas \"aven linsens br\"annvidd)
\item f \"ar avst\aa ndet mellan lins och f\"orem\aa l
\item f \"ar avst\aa ndet mellan lins och bild
\end{itemize}
    \item \alert{Linj\"ara f\"orstoringen } ges av $G=\frac{b}{a}$
    \item Br\"annvidden \"ar positiv f\"or de konvexa samlingslinserna och negativ f\"or de konkava spridningslinserna
    \item Bildavst\aa ndet(b) \"ar positivt f\"or reella(verkliha) bilder och negativt f\"or virtuella
    \item \alert{Dioptritalet} ges av $D=\frac{1}{f}$, d\"ar f uttrycks i meter
    \item 
  \end{itemize}



\end{frame}

\end{document}